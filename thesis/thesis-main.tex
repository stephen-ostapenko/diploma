\documentclass[a4paper,article,14pt]{extarticle}

% Подключаем главный пакет со всем необходимым
\usepackage{spbudiploma}

% Пакеты по желанию (самые распространенные)
% Хитрые мат. символы
\usepackage{euscript}
% Таблицы
\usepackage{longtable}
\usepackage{makecell}
% Картинки (можно вставлять даже pdf)
\usepackage[pdftex]{graphicx}

\usepackage{amsthm,amssymb, amsmath}
\usepackage{textcomp}

\usepackage{minted} % для примеров кода (требует параметра -shell-escape)
\usemintedstyle{bw}

% ===============================================

\renewcommand{\le}{\leqslant}
\renewcommand{\ge}{\geqslant}

% ===============================================


\begin{document}

% Титульник в файле titlepage.tex
\newgeometry{left=30mm, top=20mm, right=15mm, bottom=20mm, nohead, nofoot}
\begin{titlepage}
\begin{center}

\textbf{Санкт--Петербургский государственный университет}\\
\textbf{Факультет математики и компьютерных наук}


\vspace{35mm}

\textbf{\textit{\large Остапенко Степан Сергеевич}} \\[8mm]
% Название
\textbf{\large Выпускная квалификационная работа}\\[3mm]
\textbf{\textit{\large Проверка выполнимости SMT-формул\\с помощью нейронных сетей}}

\vspace{20mm}
Уровень образования: бакалавриат\\
Направление 01.03.02 «Прикладная математика и информатика»\\
Основная образовательная программа СВ.5156.2020\\
«Современное программирование»\\[15mm]

% Научный руководитель, рецензент
\begin{flushright}
\begin{minipage}[t]{0.55\textwidth}
{Научный руководитель:} \\
доцент, \\
Факультет математики и \\
компьютерных наук СПбГУ, \\
к.ф.-м.н. Шалымов Дмитрий Сергеевич

\vspace{10mm}

{Рецензент:} \\
ведущий инженер ключевых проектов, \\
ООО «Техкомпания Хуавэй», \\
Ковальчук Сергей Валерьевич
\end{minipage}
\end{flushright}

\vspace{8mm}

{Санкт-Петербург}
\par{\the\year{} г.}
\end{center}
\end{titlepage}
\restoregeometry
\addtocounter{page}{1}


% Содержание
\tableofcontents
\pagebreak

% Также существуют особые главы - это "Введение", "Постановка задачи", "Обзор литературы", "Выводы", "Заключение". Они обязательно присутствуют, не нумеруются и должны идти в том порядке, в каком они идут в примере. 

\specialsection{Введение}

Введение широко представляет предметную область работы, указывает на место работы в научном или технологическом контексте.

\specialsection{Постановка задачи}

В постановке задачи коротко (по пунктам) указывается, что необходимо сделать в рамках работы.


\section{глава}

текст

текст \cite{test}

\specialsection{Заключение}

Заключение должно подводить итоги работы и содержать информацию о полученных в рамках работы результатах.


% Аргумент {1} ниже включает переопределенный стиль с выравниванием слева
\begin{thebibliography}{1}

\bibitem{smt-logics-picture} SMT-LIB 2 Logics. URL: \url{https://smt-lib.org/logics.shtml} (дата обр. 13.05.2024).

\bibitem{z3-paper} Leonardo de Moura and Nikolaj Bjørner. <<Z3: an efficient SMT solver>>. In proceedings of \textit{Tools and Algorithms for the Construction and Analysis of Systems} (TACAS '2008), pp. 337--340. Springer, 2008.

\bibitem{cvc5-paper} Haniel Barbosa, Clark W. Barrett, Martin Brain, Gereon Kremer, Hanna Lachnitt, Makai Mann, Abdalrhman Mohamed, Mudathir Mohamed, Aina Niemetz, Andres N{\"{o}}tzli, Alex Ozdemir, Mathias Preiner, Andrew Reynolds, Ying Sheng, Cesare Tinelli and Yoni Zohar; Dana Fisman (ed.) and Grigore Rosu (ed.). <<cvc5: {A} Versatile and Industrial-Strength {SMT} Solver>>. In proceedings of \textit{Tools and Algorithms for the Construction and Analysis of Systems} (TACAS '2022), part~{I}, pp. 415--442. Springer, 2022.

\bibitem{yices2-paper} Bruno Dutertre; Armin Biere (ed.) and Roderick Bloem (ed.). <<Yices 2.2>>. In proceedings of \textit{Computer-Aided Verification} (CAV '2014), pp. 737--744. Springer, 2014.

\bibitem{bitwuzla-paper} Aina Niemetz and Mathias Preiner; Constantin Enea (ed.) and Akash Lal (ed.). <<Bitwuzla>>. In proceedings of \textit{Computer-Aided Verification} (CAV '2023), part~{II}, pp. 3--17. Springer, 2023.

\bibitem{smt-comp-paper} Clark Barrett, Leonardo de Moura and Aaron Stump. <<SMT-COMP: Satisfiability modulo theories competition>>. In proceedings of \textit{Computer-Aided Verification} (CAV '2005), pp. 503--516. Springer, 2005.

\bibitem{smt-comp-website} SMT-COMP. URL: \url{https://smt-comp.github.io/} (дата обр. 15.05.2024).

\bibitem{symbex-intro-paper} James C. King. <<Symbolic execution and program testing>>. In \textit{Communications of the ACM}, July 1976, volume 19, number 7, pp. 385--394. Association for Computing Machinery, 1976.

\bibitem{klee-website} KLEE Symbolic Execution Engine. URL: \url{https://klee-se.org/} (дата обр. 15.05.2024).

\bibitem{klee-paper} Cristian Cadar, Daniel Dunbar and Dawson Engler. <<KLEE: Unassisted and Automatic Generation of High-Coverage Tests for Complex Systems Programs>>. In proceedings of \textit{the 8th USENIX conference on Operating systems design and implementation} (OSDI '2008), pp. 209--224. USENIX Association, 2008.

\bibitem{fastsmt-paper} Mislav Balunovic, Pavol Bielik and Martin T. Vechev. <<Learning to Solve SMT Formulas>>. \textit{Advances in Neural Information Processing Systems 31} (2018).

\bibitem{gnn-for-scheduling-paper} Jan H\r{u}la, David Moj\v{z}\'{\i}\v{s}ek and Mikol\'{a}\v{s} Janota. <<Graph Neural Networks for Scheduling of SMT Solvers>>. \textit{IEEE 33rd International Conference on Tools with Artificial Intelligence} (ICTAI '2021), pp. 447--451. IEEE, 2021.

\end{thebibliography}


\end{document}
