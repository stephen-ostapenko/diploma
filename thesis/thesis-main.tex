\documentclass[a4paper,article,14pt]{extarticle}

% Подключаем главный пакет со всем необходимым
\usepackage{spbudiploma}

% Пакеты по желанию (самые распространенные)
% Хитрые мат. символы
\usepackage{euscript}
% Таблицы
\usepackage{longtable}
\usepackage{makecell}
% Картинки (можно вставлять даже pdf)
\usepackage[pdftex]{graphicx}
\usepackage{float}

\usepackage{amsthm,amssymb,amsmath}
\usepackage{textcomp}
\usepackage{ulem}

\usepackage{minted} % для примеров кода (требует параметра -shell-escape)
\usemintedstyle{tango}
\renewcommand{\theFancyVerbLine}{\normalsize \arabic{FancyVerbLine}}
\usepackage{listings}

\usepackage{hyperref}
\usepackage{comment}

% ===============================================

\renewcommand{\le}{\leqslant}
\renewcommand{\ge}{\geqslant}

\newcommand{\fakesc}[1]{\uppercase{{\footnotesize #1}}}
\renewcommand{\textsc}{\fakesc}

% ===============================================


\begin{document}

% Титульник в файле titlepage.tex
\newgeometry{left=30mm, top=20mm, right=15mm, bottom=20mm, nohead, nofoot}
\begin{titlepage}
\begin{center}

\textbf{Санкт--Петербургский государственный университет}\\
\textbf{Факультет математики и компьютерных наук}


\vspace{35mm}

\textbf{\textit{\large Остапенко Степан Сергеевич}} \\[8mm]
% Название
\textbf{\large Выпускная квалификационная работа}\\[3mm]
\textbf{\textit{\large Проверка выполнимости SMT-формул\\с помощью нейронных сетей}}

\vspace{20mm}
Уровень образования: бакалавриат\\
Направление 01.03.02 «Прикладная математика и информатика»\\
Основная образовательная программа СВ.5156.2020\\
«Современное программирование»\\[15mm]

% Научный руководитель, рецензент
\begin{flushright}
\begin{minipage}[t]{0.55\textwidth}
{Научный руководитель:} \\
доцент, \\
Факультет математики и \\
компьютерных наук СПбГУ, \\
к.ф.-м.н. Шалымов Дмитрий Сергеевич

\vspace{10mm}

{Рецензент:} \\
ведущий инженер ключевых проектов, \\
ООО «Техкомпания Хуавэй», \\
Ковальчук Сергей Валерьевич
\end{minipage}
\end{flushright}

\vspace{8mm}

{Санкт-Петербург}
\par{\the\year{} г.}
\end{center}
\end{titlepage}
\restoregeometry
\addtocounter{page}{1}


% Содержание
\tableofcontents
\pagebreak

% Также существуют особые главы - это
% "Введение",
% "Постановка задачи",
% "Обзор литературы",
% "Выводы",
% "Заключение".
% Они обязательно присутствуют, не нумеруются и должны идти в том порядке, в каком они идут в примере. 

% \specialsection{Введение}

\specialsubsection{SMT-формулы}

\underline{SMT-формулы} (от англ. \textit{Satisfiability Modulo Theories}) являются одним из наиболее важных объектов в области практических применений математической логики. Они позволяют сначала формальным образом записать разнообразные утверждения из различных предметных областей, а затем с помощью специальных вычислительных методов проверить логическую выполнимость (состоятельность) этих утверждений.

SMT-формула является некоторым расширением SAT-формулы, которое позволяет использовать не только логические переменные и связки с ними, а ещё и выражения с участием объектов из некоторого домена (например: целые числа, вещественные числа, битовые векторы, списки) и различными операциями с ними (например: числовая арифметика, битовые операции, взятие элемента списка по индексу и т. д.).

Более формально, SMT-формула является формулой в логике первого порядка, где у каждого символа переменной или функции есть некоторый заранее определенный тип (домен), а в качестве функциональных символов используются функции и операторы из разных доменов (например, те же битовые операции или взятия элемента списка по индексу). Задача проверки выполнимости такой формулы состоит в том, чтобы выяснить, можно ли для каждой свободной (не находящейся под квантором) переменной подобрать значение соответствующего ей типа так, чтобы при подстановке данных значений формула была истинна.

Чтобы стало более понятно, покажу несколько примеров.

Пример формулы:
\begin{equation}
    (x + y + z = 2) \wedge (x + 2y + 3z = 2) \wedge (x + 2y + 4z = 1),
    \label{smt-example-1}
\end{equation}

где $x$, $y$ и $z$ --- целые числа. Эта формула является выполнимой, поскольку при подстановке значений $x \leftarrow 1$, $y \leftarrow 2$, $z \leftarrow -1$ формула превращается в истинное утверждение.

Ещё пример:
\begin{equation}
    (x = 2y) \wedge (x = 2z + 1).
    \label{smt-example-2}
\end{equation}

Если потребовать, чтобы $x$, $y$ и $z$ в этой формуле были целыми числами, то формула, очевидно, не будет выполнимой (т. к. в этом случае написанное здесь обозначает высказывание \textit{$x$ является чётным и $x$ является нечётным}). Однако, если разрешить переменным быть вещественными (или хотя бы рациональными), формула станет выполнимой. Этот пример показывает, что свойство выполнимости зависит от типовых (доменных) ограничений, наложенных на переменные и функциональные символы.

Пример формулы с кванторами:
\begin{equation}
    \exists \ x, y, z: (n \ge 0) \wedge (x > 0) \wedge (y > 0) \wedge (z > 0) \wedge (x^n + y^n = z^n),
    \label{smt-example-3}
\end{equation}

где все переменные являются целыми числами. В этой формуле присутствует только одна свободная переменная --- $n$, поэтому искать подходящее значение нужно только для неё. Нетрудно заметить, что нам подходят значения $n = 1$ и $n = 2$, поэтому формула является выполнимой (но если условие $n \ge 0$ заменить на $n \ge 3$, формула станет невыполнимой согласно Великой теореме Ферма).

Пример формулы с битовыми векторами:
\begin{equation}
    (x \ge \texttt{\#b01000000}) \wedge (x + (x \ \texttt{shl}\footnote{Здесь $\texttt{shl}$ --- \textit{shift left}: $x \ \texttt{shl} \ y$ есть битовый сдвиг вектора $x$ влево на $y$ битов.} \ \texttt{\#b00000001}) + \texttt{\#b11111111} < \texttt{\#b00100000}),
    \label{smt-example-4}
\end{equation}

где переменная $x$ является битовым вектором размера 8 (фактически, это беззнаковое числом размером 1 байт). Аналогично, все константы\footnote{Вообще константы не принято выносить как отдельную сущность в формулах, поскольку их можно считать функциями от нулевого количества аргументов и, соответственно, обозначать каждую из них своим функциональным символом из логики первого порядка. Но в этой работе для удобства я буду отделять константы, содержащиеся в формулах.} в формуле тоже являются битовыми векторами размера 8 и записываются в бинарном формате. Стоит также отметить, что каждый битовый вектор можно однозначно отождествить с числом через его запись в двоичной системе счисления (например, таким образом происходит сравнение векторов на больше-меньше).

В указанном примере демонстрируется возможность использования переменных, функций и операторов из домена битовых векторов, а также то, что их семантика может отличаться от общепринятых семантик в других доменах. Так, например, все арифметические операции с битовыми векторами (в нашем случае, это два сложения) выполняются с учётом битового переполнения. Из этого, в частности, следует, что формула из примера является выполнимой, т. к. нам подходит значение $x = \texttt{\#b01100000}$ (или 96 в десятичной системе счисления):
\begin{itemize}
    \item $x = \texttt{\#b01100000}$;
    \item $x \ge \texttt{\#b01000000}$ очевидно выполняется;
    \item $x \ \texttt{shl} \ \texttt{\#b00000001} = \texttt{\#b11000000}$;
    \item $x + (x \ \texttt{shl} \ \texttt{\#b00000001}) = \texttt{\#b01100000} + \texttt{\#b11000000} = \texttt{\#b00100000}$ из-за битового переполнения;
    \item $x + (x \ \texttt{shl} \ \texttt{\#b00000001}) + \texttt{\#b11111111} = \texttt{\#b00100000} + \texttt{\#b11111111} = \texttt{\#b00011111}$ опять же из-за переполнения;
    \item $x + (x \ \texttt{shl} \ \texttt{\#b00000001}) + \texttt{\#b11111111} = \texttt{\#b00011111} < \texttt{\#b00100000}$, что и требовалось.
\end{itemize}

% todo: пример с массивами

\specialsubsection{SMT-логики}

Говоря про домены, модальности и типы данных в SMT-формулах, важно рассказать про \underline{SMT-логики} (в русскоязычной литературе их ещё часто называют теориями). Логики задают ограничения на типы переменных, констант, функций и операторов, которые можно использовать в формуле. Согласно языку логики первого порядка, в каждой формуле вне зависимости от теории можно использовать пропозициональные переменные (ложь, истина) и константы, логические связки между ними и кванторы. Возможность использования остальных конструкций уже определяется логикой.

Так, например, логика \texttt{LIA} (\textit{Linear Integer Arithmetic}) позволяет использовать только целочисленные переменные и константы сколь угодно больших по модулю значений, а также арифметические операции и операторы сравнения (больше-меньше-равно) с их участием. Вдобавок, поскольку это линейная арифметика, все термы формулы обязаны быть линейными по каждой из переменных. Примером формулы в такой логике является формула (\ref{smt-example-1}).

Ещё один пример логики --- \texttt{LRA} (\textit{Linear Real Arithmetic}) --- аналог \texttt{LIA}: устроена так же, но позволяет использовать только вещественные переменные и константы (т. е. проверить, решается ли уравнение в целых числах, с помощью этой логики не получится). Здесь уместно вспомнить формулу (\ref{smt-example-2}), которая является выполнимой в логике \texttt{LRA}, но не является выполнимой в логике \texttt{LIA}.

Нелинейные арифметики тоже реализованы в виде логик: \texttt{NIA} (\textit{Non-linear Integer Arithmetic}) и \texttt{NRA} (\textit{Non-linear Real Arithmetic}) во всём похожи на свои линейные аналоги, но не требуют линейности формулы по переменным (формула (\ref{smt-example-3})).

Битовые векторы представлены в логике \texttt{BV} (\textit{Bit Vector}). Эта логика позволяет записывать формулы с битовыми векторами произвольного фиксированного размера, используя битовые (и, или, исключающее или, отрицание, битовые сдвиги и т. д.), структурные (конкатенация двух битовых векторов, вырезание произвольного подотрезка битового вектора, циклические сдвиги и т. д.) и арифметические (сумма, разность, произведение, целочисленное деление, остаток от деления и т. д.) операции с ними, а также операторы сравнения.

Как уже было отмечено, все арифметические операции с битовыми векторами выполняются с учётом битового переполнения, а сравнения на больше-меньше осуществляются через сравнения целых чисел, двоичными записями которых являются сравниваемые векторы. Кроме того, для удобства, каждая арифметическая операция и каждое сравнение есть в двух вариантах: беззнаковом и знаковом. Знаковый вариант отличается тем, что двоичная запись числа (которая порождается битовым вектором) читается не в обычной арифметической семантике, где каждый бит обозначает очередную степень двойки, а в семантике <<дополнение до двух>>, где самая старшая степень двойки учитывается с минусом (именно таким образом хранятся знаковые целые числа в памяти современных компьютеров).

\specialsubsection{SMT-решатели}

\specialsubsection{Применения}

\specialsubsection{Символьное исполнение}


% \specialsection{Постановка задачи}

Поскольку, как уже было отмечено, SMT-решатели периодически зависают в процессе проверки очередной формулы, что негативно сказывается на времени решения прикладной задачи, хотелось бы иметь инструмент, который за достаточно разумное время мог бы посмотреть на формулу и гарантированно что-нибудь сказать про неё, например, выдать свою степень уверенности в том, что формула, на самом деле, является выполнимой.

Такой подход может быть применён на практике в случаях, когда в задаче есть сразу несколько формул, которые нужно проверить на выполнимость, и нам, в соответствии с некоторым здравым смыслом, интересно проверять только выполнимые формулы, или наоборот, только невыполнимые. Например, как уже сказано выше, в процессе символьного исполнения выполнимость формулы ограничений пути до состояния является критерием достижимости данного состояния. Каждый раз, когда мы находим достижимое состояние, система совершает прогресс\footnote{На самом деле, когда мы понимаем, что некоторое состояние недостижимо, система тоже совершает прогресс, но в большинстве случаев он является менее значительным.}, поэтому хочется как можно чаще проверять выполнимость ограничений пути в тех случаях, когда они действительно выполнимы.

В наше время задачи предсказания различных значений для сложно устроенных данных чаще всего решаются с помощью нейронных сетей. Таким образом, цель данной работы: \textit{исследовать возможность применения нейронных сетей для предсказания выполнимости SMT-формул в различных логиках}.

Отмечу также, что в поставленной формулировке задачу можно рассматривать, в том числе, как задачу ранжирования, однако для большей общности, а также с учётом того, что в открытых источниках нет никакой информации об исследованиях про способы анализа выполнимости SMT-формул исключительно с помощью нейронных сетей, я буду пытаться найти решение в более простом виде --- в виде решения задачи классификации. Тем не менее, одно из практических применений полученных моделей будет опираться на их использование в контексте ранжирования.

Для достижения поставленной цели я выделил следующие шаги:

\begin{enumerate}
    \item Подобрать датасет\footnote{Набор данных.} для обучения модели и оценки качества, а также соответствующие метрики.
    \item Исследовать существующие подходы к представлению SMT-формул в различных задачах.
    \item Исследовать различные архитектуры нейронных сетей, которые могут быть применены для решения общей задачи.
    \item Реализовать инфраструктуру для подготовки данных, обучения моделей и оценки их качества.
    \item Выполнить обучение моделей и оценить их качество.
    \item * Если качество будет удовлетворительным, можно попытаться подготовить версию модели для использования на практике\footnote{Достижение этого шага заранее не предполагается, но если его получится выполнить, будет хорошо.}.
\end{enumerate}

% todo: дописать про USVM


% \specialsection{Обзор литературы и предметной области}

Тема применения нейронных сетей в контексте решения формально-логических задач развита не особо сильно, поскольку дискретные задачи со строгими ограничениями тяжело даются сетям с известными на данный момент архитектурами. Тем не менее, несмотря на полное отсутствие подходов, аналогичных рассматриваемому в данной работе, и основанных на применении нейросетей для проверки выполнимости формул в логике первого порядка, известны многочисленные попытки использовать нейросети в качестве помощника для алгоритма, решающего ту или иную логическую, комбинаторную или оптимизационную задачу. Про несколько таких подходов, имеющих отношение к проверке выполнимости и поиску решения для SMT-формул, я расскажу далее. Но перед этим я детально опишу устройство и архитектуру графовых нейронных сетей, поскольку на использовании данной архитектуры в значительной части основана моя работа.

\specialsubsection{Устройство GNN} \label{gnn-architecture}

Графовые нейронные сети\footnote{Далее по тексту будем называть их GNN (англ. \textit{Graph Neural Networks}).}, наряду со свёрточными, рекуррентными и т. д., являются одним из способов решать задачу машинного обучения, учитывая специфику данных. Так как графы отображают взаимоотношения между объектами некоторого множества, то и процесс построения модели, в данном случае, учитывает локальность и связи между произвольными объектами из предметной области.

Впервые архитектура GNN в общем виде была описана ещё в 2009 году в статье \cite{gnn-intro-paper}, однако широкую известность получила только в 2017 году после её применения для предсказания квантовых чисел в вычислительной органической химии \cite{gnn-quantum-chemistry-paper}. В основе вычислений лежит процесс \textit{передачи сообщений} между вершинами графа, который изображён\footnote{Подобное вычисление происходит для каждой вершины на каждой итерации процесса \textit{передачи сообщений}.} на рис.~\ref{message-passing-nn-architecture}: в каждой вершине хранится вектор-состояние (эмбеддинг\footnote{От англ. \textit{embedding} --- в машинном обучении так называют представление некоторого объекта в многомерном векторном пространстве.}), и на каждом шаге все вершины сначала рассылают свой текущий вектор всем своим непосредственным соседям, а потом агрегируют пришедшие от соседей векторы (\textit{сообщения}) и обновляют свой вектор-состояние с учётом этого; далее эти шаги повторяются несколько раз, после чего полученные таким образом векторы-состояния используются для построения представления графа и дальнейшего решения задачи.

% todo: написать, что вектор, состояние и эмбеддинг -- это одно и то же

\begin{figure}[ht]
\begin{center}
    \includegraphics[scale=0.75]{./assets/message-passing-nn-architecture.pdf}
    \caption{\label{message-passing-nn-architecture} Схема обновления вектора-состояния вершины $v$ через векторы-сообщения от её соседей $u_1$, $u_2$, \ldots, $u_n$. Цвет (синий / красный) обозначает размерность. Более тёмный цвет вверху обозначает обновлённое состояние.}
\end{center}
\end{figure}

Согласно \cite{gnn-deep-learning-5g}, формально весь процесс устроен следующим образом:

\begin{itemize}
    \item у каждого ребра $e$ есть набор параметров $s_e \in \mathbb{R}^m$;
    \item в начале вычислений в каждой вершине $v$ содержится содержится вектор (состояние) $x_v^{(0)} \in \mathbb{R}^n$ с некоторой информацией;
    \item производится несколько итераций \textit{передачи сообщений}, на $t$-й из них производится обновление векторов в вершинах по правилу, которое описывается формулой~(\ref{gnn-state-update-rule});
    \item после всех итераций предполагается, что вычисленные векторы (состояния) образуют некоторое представление вершины / графа, поэтому их можно использовать в качестве параметров\footnote{Будем также называть их признаками или фичами (от англ. \textit{feature}).} вершины / графа, подавая в какую-нибудь MLP\footnote{Многослойный перцептрон (от англ. \textit{Multi-Layer Perceptron}) --- последовательная комбинация из полносвязных линейных слоёв и нелинейных слоёв активации.}-сеть.
\end{itemize}

\begin{equation} \label{gnn-state-update-rule}
    x_v^{(t)} = \gamma^{(t)} \left(x_v^{(t - 1)}, \bigoplus_{u \in \mathcal{N}(v)} \phi^{(t)} \left(x_v^{(t - 1)}, x_u^{(t - 1)}, s_{e(u \, \to \, v)} \right) \right)
\end{equation}

Обозначения в формуле~(\ref{gnn-state-update-rule}):

\begin{itemize}
    \item $x_v^{(t)} \in \mathbb{R}^n$ --- описанные выше эмбеддинги вершин;
    \item $e(u \, \to \, v)$ --- ребро из вершины $u$ в вершину $v$, а $s_{e(u \, \to \, v)} \in \mathbb{R}^m$ --- его параметры;
    \item $\mathcal{N}(v)$ --- окрестность вершины $v$;
    \item $\phi^{(t)}: \mathbb{R}^n \times \mathbb{R}^n \times \mathbb{R}^m \to \mathbb{R}^k$ --- функция создания сообщения по состояниям вершин и параметрам ребра; может быть представлена аффинным преобразованием, композицией линейных слоёв и нелинейных активаций, какими-нибудь сложными LSTM\footnote{Долгая краткосрочная память (от англ. \textit{Long Short-Term Memory}) --- известная модификация рекуррентной нейронной сети.} или GRU\footnote{Управляемый рекуррентный блок (от англ. \textit{Gated Recurrent Unit}) --- ещё одна известная модификация рекуррентной нейронной сети.}-слоями или просто любым дифференцируемым преобразованием с обучаемыми или необучаемыми параметрами;
    \item $\bigoplus: \mathbb{R}^k \to \mathbb{R}^k$ --- функция агрегации, например: сумма, среднее или взвешенная сумма с учётом механизма внимания;
    \item $\gamma^{(t)}: \mathbb{R}^n \times \mathbb{R}^k \to \mathbb{R}^n$ --- функция обновления вектора-состояния (эмбеддинга) вершины; аналогична $\phi^{(t)}$.
\end{itemize}

В качестве примера построения сети по такой схеме можно рассмотреть Graph Convolutional Network \cite{gcn-conv-paper}, самую популярную GNN-архитектуру. В её случае мы имеем:

\begin{itemize}
    \item $\phi^{(t)} = W^T \cdot x_u^{(t - 1)}$;
    \item $\bigoplus = \sum \limits_{u \in \mathcal{N}(v) \cup \{v\}} \dfrac{1}{\sqrt{|\mathcal{N}(v)|} \cdot \sqrt{|\mathcal{N}(u)|}} \cdot \phi^{(t)} \left(x_v^{(t - 1)}, x_u^{(t - 1)}, s_{e(u \, \to \, v)} \right)$;
    \item $\gamma^{(t)} = \bigoplus \limits_{u \in \mathcal{N}(v)} \phi^{(t)} \left(x_v^{(t - 1)}, x_u^{(t - 1)}, s_{e(u \, \to \, v)} \right) + b$;
\end{itemize}

\noindent где матрица весов $W$ и вектор-сдвиг $b$ являются выучиваемыми параметрами. Итого получаем:

\begin{equation}
    x_v^{(t)} = \sum \limits_{u \in \mathcal{N}(v) \cup \{v\}} \dfrac{1}{\sqrt{|\mathcal{N}(v)|} \cdot \sqrt{|\mathcal{N}(u)|}} \cdot \left(W^T \cdot x_u^{(t - 1)} \right) + b
\end{equation}

В настоящий момент GNN широко применяются для решения задач вычислительной физики, химии и биологии, построения рекомендательных систем, прогнозирования трафика на дорогах, распознавания объектов, синтеза текстов и речи \cite{gnn-global-overview} \cite{gnn-deep-learning-5g}.

% todo: написать, что такая штука совмещает в себе CNN и RNN

\specialsubsection{FastSMT}

В статье \cite{fastsmt-paper} рассматривается идея повышения эффективности SMT-решателя за счёт оптимизации его работы на формулах, возникающих в задачах из фиксированной предметной области.

Решатель в процессе своей работы применяет к формуле большое количество разных семантически эквивалентных преобразований\footnote{В статьях и документациях их называют тактиками.}, чтобы привести её к некоторому удобному для себя виду, в котором он сможет либо достаточно быстро найти решение для формулы и доказать, что оно подходит (тем самым доказав, что формула является выполнимой), либо достаточно быстро доказать, что указанный в формуле набор ограничений невозможно выполнить, и формула является невыполнимой. Примеры тактик: замена переменной, нормализация границ неравенства, bit-blasting (представление переменной в виде набора пропозициональных значений), переписывание условного оператора через конъюнкцию импликаций и т. д.. Упомянутый ранее решатель Z3 \cite{z3-paper} поддерживает более ста подобных операций.

Цепочка преобразований, которые решатель производит над формулой, формируется согласно некоторой стратегии, которая учитывает самые разнообразные параметры (признаки) формулы: от её размера и количества свободных переменных до некоторого приближения абстрактно-синтаксического дерева формулы. Сама стратегия, в данном случае, больше всего похожа на решающее дерево, у которого в вершинах стоят ограничения на очередные параметры формулы, а в каждом листе находится тактика, которую стоит применить в случае, когда параметры соответствуют пути в этот лист.

Базовую стратегию, которая используется в решателе по-умолчанию, подбирают его создатели, причём они делают это так, чтобы его средняя скорость работы на произвольной формуле была как можно лучше. В результате, решатель работает более стабильно (т. е. реже уходит в экспоненциальный перебор возможных решений, когда на вход подают сложную формулу), но скорость нахождения ответа в более простых случаях из-за этого проседает.

В то же время, большинство популярных современных инструментов для решения SMT-формул поддерживают возможность использования специфических стратегий, которые можно построить, в том числе, самостоятельно. Помимо этого, есть некоторая интуиция, что если мы будем рассматривать только формулы из некоторой фиксированной практической задачи, то все эти формулы будут обладать некоторой спецификой, знание о которой сможет существенно помочь в проверке их выполнимости. На этой интуиции, а также на возможности использовать собственные стратегии при решении формул строится подход, описанный в рассматриваемой статье.

Главная идея состоит в следующем: набор признаков формулы можно рассматривать как состояние некоторой среды, применение очередной тактики можно рассматривать как некоторое действие, совершаемое в этой среде, а награду за последовательность применённых тактик можно определить как суммарное время работы решателя на заданной формуле с использованием этой последовательности (разумеется, взятое с минусом), либо какому-нибудь штрафному значению в случае, если решатель не справляется с формулой. Таким образом, задача поиска оптимальной стратегии применения различных тактик хорошо представляется в виде задачи обучения с подкреплением, где средой является произвольная формула из некоторой фиксированной задачи. Поэтому предлагается выучивать модель (политику), которая по текущему состоянию среды (формулы) будет выдавать оптимальную тактику и параметры для неё.

\begin{figure}[ht]
\begin{center}
    \includegraphics[scale=0.65]{./assets/fastsmt-nn-policy.pdf}
    \caption{\label{fastsmt-architecture} Архитектура модели. Картинка взята из статьи \cite{fastsmt-paper}.}
\end{center}
\end{figure}

Архитектура модели изображена на рис.~\ref{fastsmt-architecture}. На вход подаются признаки формулы (Formula Measures) и её синтаксическое представление, закодированное в виде bag-of-words или $n$-граммной модели (Formula Representation). Вдобавок к этому, для каждой тактики выучивается эмбеддинг, который хранит в себе семантическую информацию про неё. Эмбеддинги тактик, которые можно применить в данный момент, также подаются на вход модели (Prior Actions). На выходе у модели распределение на тактиках, которые стоит применить в данный момент, и значения параметров для них. Сразу отмечу, что, несмотря на то, что модель выдаёт распределение на всех доступных для применения в данный момент тактиках, за одно действие применяется только одна из них --- наиболее вероятная.

Построенная таким образом модель обучается на наборе формул взятом из некоторой задачи (в статье это были различные известные бенчмарки для решателей, на каждом из которых обучалась и оценивалась отдельная модель) с помощью метода, похожего на кросс-энтропийный метод. После этого с помощью техник семплирования строятся разнообразные статистики, отражающие выученную моделью политику, а уже на них обучается решающее дерево, выбирающее нужную тактику по текущей формуле, которое впоследствии превращается в стратегию, которую можно загрузить в SMT-решатель.

Авторы статьи заявляют, что, согласно проведённым экспериментам, предложенный подход позволяет успешно решать на 17\% больше формул при десятисекундном ограничении по времени, а также даёт прирост производительности вплоть до стократного ускорения на некоторых формулах (при этом, замедления при решении других формул не наблюдается). Тем не менее, у такого подхода есть существенный недостаток: для каждой новой задачи нужно сначала самостоятельно собирать данные, потом запускать тяжеловесный процесс обучения, а после этого самому синтезировать стратегию. Это требует довольно больших вычислительных мощностей.

\specialsubsection{GNN for Scheduling of SMT Solvers} \label{gnn-for-scheduling-of-smt-solvers}

В статье \cite{gnn-for-scheduling-paper} рассматривается более общий подход к задаче: известно, что в основе работы разных SMT-решателей лежат разные алгоритмы и эвристики, поэтому их производительность при решении разных формул может значительно варьироваться; в связи с этим, давайте просто обучим модель, которая по формуле будет предсказывать, за какое время тот или иной решатель сможет проверить её на выполнимость, а далее перед каждой проверкой будем выбирать решатель, для которого модель предсказывает наименьшее время работы.

Данная статья интересна тем, что в ней рассматривается более точное представление формулы, подаваемой модели на вход. Вместо сбора разных статистик с формулы и превращения их в признаки авторы предлагают использовать графовую нейронную сеть (GNN), которая задействует абстрактно-синтаксическое дерево\footnote{Далее в тексте будет использоваться аббревиатура AST (англ. \textit{Abstract-Syntax Tree}).} формулы в качестве графа вычислений (рис.~\ref{gnn-for-scheduling-architecture}).

\begin{figure}[ht]
\begin{center}
    \includegraphics[scale=0.24]{./assets/gnn-for-scheduling-architecture.png}
    \caption{\label{gnn-for-scheduling-architecture} Использование AST формулы в качестве графа вычислений для GNN. Картинка взята из статьи \cite{gnn-for-scheduling-paper}.}
\end{center}
\end{figure}

Чтобы ускорить работу, вершины дерева, отвечающие за эквивалентные выражения, склеиваются в одну, так что формула, на самом деле, превращается в ориентированный ациклический граф. Помимо этого, чтобы расширить распространение эмбеддинга с информацией, сохранённой в каждой вершине, к каждому ребру такого графа добавляется обратное ребро. Таким образом, информация начинает распространяться сразу во всех направлениях.

\begin{figure}[ht]
\begin{center}
    \includegraphics[scale=0.25]{./assets/gnn-for-scheduling-process.png}
    \caption{\label{gnn-for-scheduling-process} Схема использования GNN для получения информации о формуле и предсказания ответа. Картинка взята из статьи \cite{gnn-for-scheduling-paper}.}
\end{center}
\end{figure}

Далее каждая вершина (переменная, константа или операция) кодируется с помощью техники one-hot-encoding, после чего векторы с полученными таким образом значениями отправляются в GNN, на выходе у которой находится слой, собирающий эмбеддинги со всех вершин графа и предсказывающий для каждого SMT-решателя время, за которое он может выдать ответ на данной формуле. Для лучшего понимания, схема процесса изображена на рис~\ref{gnn-for-scheduling-process}.

Авторы отмечают, что подход может быть применён для решения почти для любой SMT-формулы, и утверждают, что им удалось добиться ускорения процесса на 8--900\% на разных тестах.

\specialsubsection{NeuroSAT} \label{neurosat}

Тем не менее, попытки научиться решать формально-логическую задачу, пользуясь исключительно нейронными сетями, тоже присутствуют. Пионером в этой области стала работа \cite{neurosat-paper}, в которой исследовалось применение GNN для задачи SAT. Была предложена следующая графовая архитектура: для каждой переменной, для отрицания каждой переменной и для каждого конъюнкта формулы заводится по вершине; далее рёбра проводятся между парами из литерала (переменными или их отрицаниями) и конъюнкта, если литерал входит в состав конъюнкта, а также между парами противоположных литералов (теми, которые является отрицаниями друг друга). Пример построения такого графа изображён на рис.~\ref{neurosat-mpnn}.

\begin{figure}[ht]
\begin{center}
    \includegraphics[scale=0.25]{./assets/neurosat-mpnn.png}
    \caption{\label{neurosat-mpnn} Граф, построенный для формулы $(x_1 \vee x_2) \wedge (\overline{x_1} \vee \overline{x_2})$. Пункты (a) и (b) отражают шаги построения, описанные в статье; на деле же граф выглядит как их объединение. Картинка взята из статьи \cite{neurosat-paper}.}
\end{center}
\end{figure}

Во время вычислений каждая вершина накапливает все пришедшие к ней сообщения с помощью LSTM-слоя. После каждого шага передачи сообщений по графу каждая вершина-литерал генерирует собственное предсказание по поводу того, является ли данная формула выполнимой (внутреннее состояние LSTM отправляется в многослойный классификатор (MLP), который предсказывает одно число --- степень уверенности вершины в том, что формула выполнима).

\begin{figure}[ht]
\begin{center}
    \includegraphics[scale=0.24]{./assets/neurosat-voting.png}
    \caption{\label{neurosat-voting} Граф, построенный для формулы $(x_1 \vee x_2) \wedge (\overline{x_1} \vee \overline{x_2})$. Пункты (a) и (b) отражают шаги построения, описанные в статье; на деле же граф выглядит как их объединение. Картинка взята из статьи \cite{neurosat-paper}.}
\end{center}
\end{figure}

% todo: дописать neurosat

% эмбеддинги
% дописать про GatedGraphConv https://pytorch-geometric.readthedocs.io/en/latest/generated/torch_geometric.nn.conv.GatedGraphConv.html#torch_geometric.nn.conv.GatedGraphConv
% слой голосования
% итерации
% результаты
% случайные формулы

% \hrule

% todo: дописать про другие работы
% \specialsubsection{Другие работы}

% mach smt
% alpha geometry
% Learning the Satisfiability of Pseudo-Boolean Problem with Graph Neural Networks
% https://ruoyuwang.me/bar2019/pdfs/bar2019-final80.pdf
% https://www.semanticscholar.org/paper/Algorithm-selection-for-SMT-Scott-Niemetz/aff1afe03f8e2f636b972add9b03ac59f6c34223
% https://www.semanticscholar.org/paper/On-EDA-Driven-Learning-for-SAT-Solving-Li-Shi/5c4bb681fe5cb159b0d577b784fa52c952871e17
% https://www.semanticscholar.org/paper/SATformer%3A-Transformer-Based-UNSAT-Core-Learning-Shi-Li/06547a615390ba14d37c684136c30a4ac559d610
% https://www.semanticscholar.org/paper/NeuroBack%3A-Improving-CDCL-SAT-Solving-using-Graph-Wang-Hu/a613142147ef740b2daf1265e23606c80d1c2bd2
% https://www.semanticscholar.org/paper/Synthesizing-Smart-Solving-Strategy-for-Symbolic-Chen-Chen/c7f8cd87ae269bb8b85eeb627005b7884a81bee0
% Combinatorial Optimization with Graph Convolutional Networks and Guided Tree Search https://arxiv.org/abs/1810.10659
% Andrew W Senior, Richard Evans, John Jumper, James Kirkpatrick, Laurent Sifre, Tim Green, Chongli Qin, Augustin Žídek, Alexander WR Nelson, Alex Bridgland, et al. Improved protein structure prediction using potentials from deep learning. Nature, 577(7792):706–710, 2020.


% датасет
% * smt-comp
% * usvm
% * метрики
% roberta
% gnn
% результаты

% todo: написать про то, как работает message passing neural network
% todo:
% * квантизация
% * аугментации

\section{Датасет}

\section{Текстовый подход}

\section{Графовый подход}

\section{Результаты}


% \specialsection{Заключение}

В рамках данной работы была предпринята попытка создать нейронную сеть для предсказания выполнимости SMT-формулы: собрать подходящий датасет, реализовать архитектуру нейронной сети, провести ряд экспериментов с обучением и валидацией моделей и сделать выводы.

Я считаю, что сформулированная в работе задача выполнена --- основные шаги для решения поставленной поставленной задачи, описанные в соответствующем разделе, пройдены, и требуемое исследование проведено.

В процессе работы было выяснено, что для решения задачи в общем случае протестированных методов недостаточно, и нужно использовать более умные подходы к построению векторного представления формулы, о чём написано в главе~\ref{future-works}.

Однако, в контексте практического применения (решения формул, возникающих в процессе символьного исполнения на движке USVM) данная задача может быть решена с относительно хорошими результатами при помощи более-менее базовых методов (таблицы \ref{usvm-train-ds-val-results}, \ref{usvm-val-results-roc-auc-avg-prec} и \ref{usvm-val-results-precs-at-recall}). Скорее всего, это происходит из-за того, что в практической задаче возникают очень специфические формулы, которые значительно отличаются от общего случая в лице SMT-COMP, и для которых гораздо проще предсказывать ответ из-за того, что модель привязывается на некоторые особенности возникающих формул. Я думаю, что перспектива практического использования полученных результатов во многом может опираться на этот факт.

Также в процессе исследования было протестировано несколько возможных архитектур нейронной сети (разделы \ref{sage-conv-desc} и \ref{transformer-conv-desc}).

Сравнение полученного решения с аналогами не проводилось, поскольку таковых на данный момент не существует (я не считаю аналогом модель \hyperref[neurosat]{\underline{NeuroSAT}} \cite{neurosat-paper}, так как в той задаче существенно отличаются ограничения и ожидаемый результат).

Проведённое исследование можно существенно расширять и дополнять. О методах и возможных путях развития написано в главе~\ref{future-works}.


% Аргумент {1} ниже включает переопределенный стиль с выравниванием слева
\begin{thebibliography}{1}

\bibitem{smt-logics-picture} SMT-LIB 2 Logics. URL: \url{https://smt-lib.org/logics.shtml} (дата обр. 13.05.2024).

\bibitem{z3-paper} Leonardo de Moura and Nikolaj Bjørner. <<Z3: an efficient SMT solver>>. In proceedings of \textit{Tools and Algorithms for the Construction and Analysis of Systems} (TACAS '2008), pp. 337--340. Springer, 2008.

\bibitem{cvc5-paper} Haniel Barbosa, Clark W. Barrett, Martin Brain, Gereon Kremer, Hanna Lachnitt, Makai Mann, Abdalrhman Mohamed, Mudathir Mohamed, Aina Niemetz, Andres N{\"{o}}tzli, Alex Ozdemir, Mathias Preiner, Andrew Reynolds, Ying Sheng, Cesare Tinelli and Yoni Zohar; Dana Fisman (ed.) and Grigore Rosu (ed.). <<cvc5: {A} Versatile and Industrial-Strength {SMT} Solver>>. In proceedings of \textit{Tools and Algorithms for the Construction and Analysis of Systems} (TACAS '2022), part~{I}, pp. 415--442. Springer, 2022.

\bibitem{yices2-paper} Bruno Dutertre; Armin Biere (ed.) and Roderick Bloem (ed.). <<Yices 2.2>>. In proceedings of \textit{Computer-Aided Verification} (CAV '2014), pp. 737--744. Springer, 2014.

\bibitem{bitwuzla-paper} Aina Niemetz and Mathias Preiner; Constantin Enea (ed.) and Akash Lal (ed.). <<Bitwuzla>>. In proceedings of \textit{Computer-Aided Verification} (CAV '2023), part~{II}, pp. 3--17. Springer, 2023.

\bibitem{smt-comp-paper} Clark Barrett, Leonardo de Moura and Aaron Stump. <<SMT-COMP: Satisfiability modulo theories competition>>. In proceedings of \textit{Computer-Aided Verification} (CAV '2005), pp. 503--516. Springer, 2005.

\bibitem{smt-comp-website} SMT-COMP. URL: \url{https://smt-comp.github.io/} (дата обр. 15.05.2024).

\bibitem{symbex-intro-paper} James C. King. <<Symbolic execution and program testing>>. In \textit{Communications of the ACM}, July 1976, volume 19, number 7, pp. 385--394. Association for Computing Machinery, 1976.

\bibitem{klee-website} KLEE Symbolic Execution Engine. URL: \url{https://klee-se.org/} (дата обр. 15.05.2024).

\bibitem{klee-paper} Cristian Cadar, Daniel Dunbar and Dawson Engler. <<KLEE: Unassisted and Automatic Generation of High-Coverage Tests for Complex Systems Programs>>. In proceedings of \textit{the 8th USENIX conference on Operating systems design and implementation} (OSDI '2008), pp. 209--224. USENIX Association, 2008.

\bibitem{gnn-intro-paper} Franco Scarselli, Marco Gori, Ah Chung Tsoi, Markus Hagenbuchner and Gabriele Monfardini. <<The Graph Neural Network Model>>. In \textit{IEEE Transactions on Neural Networks}, vol. 20, no. 1, pp. 61--80, Jan. 2009. IEEE, 2009.

\bibitem{gnn-quantum-chemistry-paper} Justin Gilmer, Samuel S. Schoenholz, Patrick F. Riley, Oriol Vinyals and George E. Dahl. <<Neural Message Passing for Quantum Chemistry>>. \textit{ArXiv abs:1704.01212} (2017).

\bibitem{gnn-deep-learning-5g} Michael M. Bronstein, Joan Bruna, Taco Cohen and Petar Veli\v{c}kovi\'{c}. <<Geometric Deep Learning: Grids, Groups, Graphs, Geodesics, and Gauges>>. \textit{ArXiv abs:2104.13478} (2021).

\bibitem{gcn-conv-paper} Thomas N. Kipf and Max Welling. <<Semi-Supervised Classification with Graph Convolutional Networks>>. \textit{ArXiv abs:1609.02907} (2017).

\bibitem{gnn-global-overview} Jie Zhou, Ganqu Cui, Shengding Hu, Zhengyan Zhang, Cheng Yang, Zhiyuan Liu, Lifeng Wang, Changcheng Li and Maosong Sun. <<Graph Neural Networks: A Review of Methods and Applications>>. \textit{ArXiv abs:1812.08434} (2021).

\bibitem{fastsmt-paper} Mislav Balunovic, Pavol Bielik and Martin T. Vechev. <<Learning to Solve SMT Formulas>>. \textit{Advances in Neural Information Processing Systems 31} (2018).

\bibitem{gnn-for-scheduling-paper} Jan H\r{u}la, David Moj\v{z}\'{\i}\v{s}ek and Mikol\'{a}\v{s} Janota. <<Graph Neural Networks for Scheduling of SMT Solvers>>. \textit{IEEE 33rd International Conference on Tools with Artificial Intelligence} (ICTAI '2021), pp. 447--451. IEEE, 2021.

\bibitem{neurosat-paper} Daniel Selsam, Matthew Lamm, Benedikt B\"{u}nz, Percy Liang, Leonardo Mendonça de Moura and David L. Dill. <<Learning a SAT Solver from Single-Bit Supervision>>. \textit{ArXiv abs:1802.03685} (2018).

\bibitem{smt-comp-2023-benchmarks} SMT-COMP 2023 benchmarks. URL: \url{https://smt-comp.github.io/2023/benchmarks.html} (дата обр. 20.05.2024).

\bibitem{thealgorithms-github} The Algorithms --- Java. URL: \url{https://github.com/TheAlgorithms/Java} (дата обр. 20.05.2024).

\bibitem{owasp-website} OWASP benchmark. URL: \url{https://owasp.org/www-project-benchmark/} (дата обр. 21.05.2024).

\bibitem{cassandra-website} Apache Cassandra. URL: \url{https://cassandra.apache.org/_/index.html} (дата обр. 21.05.2024).

\bibitem{kafka-website} Apache Kafka. URL: \url{https://kafka.apache.org/} (дата обр. 21.05.2024).

\bibitem{spark-website} Apache Spark. URL: \url{https://spark.apache.org/} (дата обр. 21.05.2024).

\bibitem{utbot-github} UnitTestBot. URL: \url{https://github.com/UnitTestBot/UTBotJava} (дата обр. 21.05.2024).

\bibitem{zookeeper-website} Apache Zookeeper. URL: \url{https://zookeeper.apache.org/} (дата обр. 21.05.2024).

\bibitem{elasticsearch-website} Elasticsearch. URL: \url{https://www.elastic.co/elasticsearch} (дата обр. 21.05.2024).

\bibitem{hbase-website} Apache HBase. URL: \url{https://hbase.apache.org/} (дата обр. 21.05.2024).

\bibitem{guava-website} Google Guava. URL: \url{https://github.com/google/guava} (дата обр. 21.05.2024).

\bibitem{hadoop-website} Apache Hadoop. URL: \url{https://hadoop.apache.org/} (дата обр. 21.05.2024).

\bibitem{rvnn-intro-paper} Alessandro Sperduti and Antonina Starita. <<Supervised neural networks for the classification of structures>>. In \textit{IEEE Transactions on Neural Networks}, vol. 8, no. 3, pp. 714--735, May 1997. IEEE, 1997.

\bibitem{rvnn-intro-paper-2} Paolo Frasconi, Marco Gori and Alessandro Sperduti. <<A general framework for adaptive processing of data structures>>. In \textit{IEEE Transactions on Neural Networks}, vol. 9, no. 5, pp. 768--786, Sept. 1998. IEEE, 1998.

\bibitem{attention-is-all-you-need} Ashish Vaswani, Noam Shazeer, Niki Parmar, Jakob Uszkoreit, Llion Jones, Aidan N. Gomez, Lukasz Kaiser and Illia Polosukhin. <<Attention Is All You Need>>. \textit{ArXiv abs:1706.03762} (2017).

% \bibitem{negative-sampling-paper} Tomas Mikolov, Ilya Sutskever, Kai Chen, Greg Corrado and Jeffrey Dean. <<Distributed Representations of Words and Phrases and their Compositionality>>. \textit{ArXiv abs:1310.4546} (2013).

\bibitem{embeddings-for-numerical-features-paper} Yury Gorishniy, Ivan Rubachev and Artem Babenko. <<On Embeddings for Numerical Features in Tabular Deep Learning>>. \textit{Neural Information Processing Systems} (NeurIPS '2022).

\bibitem{ffm-paper-1} Matthew Tancik, Pratul P. Srinivasan, Ben Mildenhall, Sara Fridovich-Keil, Nithin Raghavan, Utkarsh Singhal, Ravi Ramamoorthi, Jonathan T. Barron and Ren Ng. <<Fourier Features Let Networks Learn High Frequency Functions in Low Dimensional Domains>>. \textit{ArXiv abs:2006.10739} (2020).

\bibitem{ffm-paper-2} Yang Li, Si Si, Gang Li, Cho-Jui Hsieh and Samy Bengio. <<Learnable Fourier Features for Multi-Dimensional Spatial Positional Encoding>>. \textit{ArXiv abs:2106.02795} (2021).

\bibitem{sage-conv-paper} William L. Hamilton, Rex Ying and Jure Leskovec. <<Inductive Representation Learning on Large Graphs>>.  \textit{ArXiv abs:1706.02216} (2017).

\bibitem{transformer-conv-paper} Yunsheng Shi, Zhengjie Huang, Shikun Feng, Hui Zhong, Wenjin Wang and Yu Sun. <<Masked Label Prediction: Unified Message Passing Model for Semi-Supervised Classification>>. \textit{ArXiv abs:2009.03509} (2021).

\bibitem{adam-paper} Diederik P. Kingma and Jimmy Ba. <<Adam: A Method for Stochastic Optimization>>. \textit{ArXiv abs:1412.6980} (2014).

\bibitem{adamw-paper} Ilya Loshchilov and Frank Hutter. <<Decoupled Weight Decay Regularization>>. \textit{ArXiv abs:1711.05101} (2017).

\bibitem{dropout-paper} Nitish Srivastava, Geoffrey E. Hinton, Alex Krizhevsky, Ilya Sutskever and Ruslan Salakhutdinov. <<Dropout: a simple way to prevent neural networks from overfitting>>. In \textit{Journal of Machine Learning Research}, vol. 15, pp. 1929--1958, June 2014. JMLR, 2014.

\bibitem{layer-norm-paper} Jimmy Lei Ba, Jamie Ryan Kiros and Geoffrey E. Hinton. <<Layer Normalization>>. \textit{ArXiv abs:1607.06450} (2016).

\bibitem{lambda-rank-paper} Christopher Burges. <<From ranknet to lambdarank to lambdamart: An overview>>. (2010).

\bibitem{triplet-loss-paper-1} Matthew Schultz and Thorsten Joachims. <<Learning a Distance Metric from Relative Comparisons>>. \textit{Neural Information Processing Systems} (NeurIPS '2003).

\bibitem{triplet-loss-paper-2} Gal Chechik, Varun Sharma, Uri Shalit and Samy Bengio. <<Large Scale Online Learning of Image Similarity Through Ranking>>. In \textit{Journal of Machine Learning Research}, vol. 11, pp. 11--14, June 2009. JMLR, 2010.

% \bibitem{usvm-diploma} Поспелов Сергей Андреевич. <<Проектирование и разработка универсальной символьной виртуальной машины>>. \textit{Бакалаврская выпускная квалификационная работа}. СПбГУ, 2023.

% \bibitem{azat} Валеев Азат Рустамович. <<Добавление признаков композиций в функцию потерь модели MusicTransformer для настраиваемой генерации музыки>>. \textit{Бакалаврская выпускная квалификационная работа}. СПбГУ, 2023.

\end{thebibliography}


\end{document}
