\specialsection{Заключение}

В рамках данной работы была предпринята попытка создать нейронную сеть для предсказания выполнимости SMT-формулы: собрать подходящий датасет, реализовать архитектуру нейронной сети, провести ряд экспериментов с обучением и валидацией моделей и сделать выводы.

Я считаю, что сформулированная в работе задача выполнена --- основные шаги для решения поставленной поставленной задачи, описанные в соответствующем разделе, пройдены, и требуемое исследование проведено.

В процессе работы было выяснено, что для решения задачи в общем случае протестированных методов недостаточно, и нужно использовать более умные подходы к построению векторного представления формулы, о чём написано в главе~\ref{future-works}.

Однако, в контексте практического применения (решения формул, возникающих в процессе символьного исполнения на движке USVM) данная задача может быть решена с относительно хорошими результатами при помощи более-менее базовых методов (таблицы \ref{usvm-train-ds-val-results}, \ref{usvm-val-results-roc-auc-avg-prec} и \ref{usvm-val-results-precs-at-recall}). Скорее всего, это происходит из-за того, что в практической задаче возникают очень специфические формулы, которые значительно отличаются от общего случая в лице SMT-COMP, и для которых гораздо проще предсказывать ответ из-за того, что модель привязывается на некоторые особенности возникающих формул. Я думаю, что перспектива практического использования полученных результатов во многом может опираться на этот факт.

Также в процессе исследования было протестировано несколько возможных архитектур нейронной сети (разделы \ref{sage-conv-desc} и \ref{transformer-conv-desc}).

Сравнение полученного решения с аналогами не проводилось, поскольку таковых на данный момент не существует (я не считаю аналогом модель \hyperref[neurosat]{\underline{NeuroSAT}} \cite{neurosat-paper}, так как в той задаче существенно отличаются ограничения и ожидаемый результат).

Проведённое исследование можно существенно расширять и дополнять. О методах и возможных путях развития написано в главе~\ref{future-works}.
