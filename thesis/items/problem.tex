\specialsection{Постановка задачи}

Поскольку, как уже было отмечено, SMT-решатели периодически зависают в процессе проверки очередной формулы, что негативно сказывается на времени решения прикладной задачи, хотелось бы иметь инструмент, который за достаточно разумное время мог бы посмотреть на формулу и гарантированно что-нибудь сказать про неё, например, выдать свою степень уверенности в том, что формула, на самом деле, является выполнимой.

Такой подход может быть применён на практике в случаях, когда в задаче есть сразу несколько формул, которые нужно проверить на выполнимость, и нам, в соответствии с некоторым здравым смыслом, интересно проверять только выполнимые формулы, или наоборот, только невыполнимые. Например, как уже сказано выше, в процессе символьного исполнения выполнимость формулы ограничений пути до состояния является критерием достижимости данного состояния. Каждый раз, когда мы находим достижимое состояние, система совершает прогресс\footnote{На самом деле, когда мы понимаем, что некоторое состояние недостижимо, система тоже совершает прогресс, но в большинстве случаев он является менее значительным.}, поэтому хочется как можно чаще проверять выполнимость ограничений пути в тех случаях, когда они действительно выполнимы.

В наше время задачи предсказания различных значений для сложно устроенных данных чаще всего решаются с помощью нейронных сетей. Таким образом, цель данной работы: \textit{исследовать возможность применения нейронных сетей для предсказания выполнимости SMT-формул в различных логиках}.

Отмечу также, что в поставленной формулировке задачу можно рассматривать, в том числе, как задачу ранжирования, однако для большей общности, а также с учётом того, что в открытых источниках нет никакой информации об исследованиях про способы анализа выполнимости SMT-формул исключительно с помощью нейронных сетей, я буду пытаться найти решение в более простом виде --- в виде решения задачи классификации. Тем не менее, одно из практических применений полученных моделей будет опираться на их использование в контексте ранжирования.

Для достижения поставленной цели я выделил следующие шаги:

\begin{enumerate}
    \item Подобрать датасет\footnote{Набор данных.} для обучения модели и оценки качества, а также соответствующие метрики.
    \item Исследовать существующие подходы к представлению SMT-формул в различных задачах.
    \item Исследовать различные архитектуры нейронных сетей, которые могут быть применены для решения общей задачи.
    \item Реализовать инфраструктуру для подготовки данных, обучения моделей и оценки их качества.
    \item Выполнить обучение моделей и оценить их качество.
    \item * Если качество будет удовлетворительным, можно попытаться подготовить версию модели для использования на практике\footnote{Достижение этого шага заранее не предполагается, но если его получится выполнить, будет хорошо.}.
\end{enumerate}

% todo: дописать про USVM
